\section{Related Work}
There are several researches of using OpenFlow to increase performance of big data analyses.
Application awareness plays a significant role to estimate traffic demand and dependencies. As in
the example of Hadoop, the Hadoop job tracker offers information about placement of map and reduce
tasks, it monitors the progress of all tasks to predict traffic demand for the network-bound shuffle
phase. \cite{programmingatruntime} \cite {pythia} provide ways to program the network at runtime and
avoid hotspots by using these information in combination of network topology and link-level
utilization.

In \cite{query}, OpenFlow is used to optimize distributed analysis queries by prioritizing network
traffics and reserving bandwidths. In addition, they match the query plans against the network
topology.  Topology aware task scheduling is also proposed in \cite{programmingatruntime}.

Our work is focused on the placement of Flink tasks by taking network topology into account. The
algorithm developed could also be used for other frameworks.

