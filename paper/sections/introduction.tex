\section{Introduction}
This paper deals with optimizing performance of parallel dataflows in Big Data Analytics utilizing
Software Defined Networking (SDN). We test if, how and by what margin advantages of SDN-based
environments over traditional ones enrich the performance of a Big Data Analytics system for use in
clusters. The significance of such software infers from the prevailing need of many businesses to
process large ever-growing volumes of data. However, such systems do not consider the topology of
the network they are deployed on. Throughput and processing time of an analytical program running on
a dedicated cluster could be improved by utilizing knowledge about the underlying network or
managing the network devices dynamically during runtime. An SDN-enabled network provides a
Controller unit or system that comprises the control logic of the network devices located on
separate computer resources resulting in the differentiation between the so called data and control
planes. It features a Southbound API for conveying information “down” to the devices in the data
plane as well as a Northbound API for communication with application and business logic “up top”.
SDN Controllers thus provide a centralized view on the network and could potentially be used to
retrieve information for utilization by the data processing platform in deployment decisions.
Furthermore the functions available for network manipulation could be invoked during runtime based
on the knowledge about currently running analytical jobs.
